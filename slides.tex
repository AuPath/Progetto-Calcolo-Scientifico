%%%%%%%%%%%%%%%%%%%%%%%%%%%%%%%%%%%%%%%%%
% Beamer Presentation
% LaTeX Template
% Version 1.0 (10/11/12)
%
% This template has been downloaded from:
% http://www.LaTeXTemplates.com
%
% License:
% CC BY-NC-SA 3.0 (http://creativecommons.org/licenses/by-nc-sa/3.0/)
%
%%%%%%%%%%%%%%%%%%%%%%%%%%%%%%%%%%%%%%%%%

%----------------------------------------------------------------------------------------
%	PACKAGES AND THEMES
%----------------------------------------------------------------------------------------

\documentclass{beamer}

\mode<presentation> {

% The Beamer class comes with a number of default slide themes
% which change the colors and layouts of slides. Below this is a list
% of all the themes, uncomment each in turn to see what they look like.

%\usetheme{default}
%\usetheme{AnnArbor}
%\usetheme{Antibes}
%\usetheme{Bergen}
%\usetheme{Berkeley}
%\usetheme{Berlin}
%\usetheme{Boadilla}
%\usetheme{CambridgeUS}
%\usetheme{Copenhagen}
%\usetheme{Darmstadt}
%\usetheme{Dresden}
%\usetheme{Frankfurt}
%\usetheme{Goettingen}
%\usetheme{Hannover}
%\usetheme{Ilmenau}
%\usetheme{JuanLesPins}
%\usetheme{Luebeck}
%\usetheme{Madrid}
%\usetheme{Malmoe}
%\usetheme{Marburg}
%\usetheme{Montpellier}
%\usetheme{PaloAlto}
%\usetheme{Pittsburgh}
%\usetheme{Rochester}
\usetheme{Singapore}
%\usetheme{Szeged}
%\usetheme{Warsaw}

% As well as themes, the Beamer class has a number of color themes
% for any slide theme. Uncomment each of these in turn to see how it
% changes the colors of your current slide theme.

%\usecolortheme{albatross}
%\usecolortheme{beaver}
%\usecolortheme{beetle}
%\usecolortheme{crane}
%\usecolortheme{dolphin}
%\usecolortheme{dove}
%\usecolortheme{fly}
%\usecolortheme{lily}
%\usecolortheme{orchid}
\usecolortheme{rose}
%\usecolortheme{seagull}
%\usecolortheme{seahorse}
%\usecolortheme{whale}
%\usecolortheme{wolverine}

%\setbeamertemplate{footline} % To remove the footer line in all slides uncomment this line
%\setbeamertemplate{footline}[page number] % To replace the footer line in all slides with a simple slide count uncomment this line

%\setbeamertemplate{navigation symbols}{} % To remove the navigation symbols from the bottom of all slides uncomment this line
}

\usepackage{graphicx} % Allows including images
\usepackage{booktabs} % Allows the use of \toprule, \midrule and \bottomrule in tables

%----------------------------------------------------------------------------------------
%	TITLE PAGE
%----------------------------------------------------------------------------------------

\title{Progetto di Metodi del Calcolo Scientifico} % The short title appears at the bottom of every slide, the full title is only on the title page

\author{Tonelli Lidia Lucrezia (m. 813114)\\
Grassi Marco (m. 830694)\\
Giudice Gianluca (m. 829664)
} % Your name
\institute[University of Milano Bicocca] % Your institution as it will appear on the bottom of every slide, may be shorthand to save space
{
University of Milano Bicocca \\ % Your institution for the title page
\medskip
}
\date{Giugno 2021} % Date, can be changed to a custom date

\begin{document}

\begin{frame}
\titlepage % Print the title page as the first slide
\end{frame}

\begin{frame}
\frametitle{Overview} % Table of contents slide, comment this block out to remove it
\tableofcontents % Throughout your presentation, if you choose to use \section{} and \subsection{} commands, these will automatically be printed on this slide as an overview of your presentation
\end{frame}

%----------------------------------------------------------------------------------------
%	PRESENTATION SLIDES
%----------------------------------------------------------------------------------------

%------------------------------------------------
\section{Metodi diretti per matrici sparse}
%------------------------------------------------

\begin{frame}
\frametitle{Cosa ci sar\'a nella presentazione}
\begin{itemize}
\item Matlab
\begin{itemize}
\item Windows
\item Linux
\end{itemize}
\item Octave
\begin{itemize}
\item Windows
\item Linux
\end{itemize}
\item Python (Numpy/Scipy)
\begin{itemize}
\item Windows
\item Linux
\end{itemize}
\end{itemize}
\end{frame}

%------------------------------------------------

\frametitle{Librerie usate - Matlab}
\begin{frame}
Matlab \'e ben documentato.\\
Come salva le matrici sparse? Come Python, ogni elemento diverso da 0 è salvato con colonna e riga\\
Come risolve il sistema lineare? Fare riassunto con la parte delle matrici sparse https://it.mathworks.com/help/matlab/ref/mldivide.html\\
Come tratta le matrici definite positive e quelle generiche? Cholesky per vedere se è definita positiva, se sì usa un metodo, se no un altro -> approfondire al link sopra
\end{frame}

\begin{frame}
\frametitle{Librerie usate - Octave}
Octave \'e mal documentato. \\
Come salva le matrici sparse? Usa il compressed column format -> si salva ogni elemento diverso da zero con il numero di riga, e per ogni colonna salva il numero nnz per quella colonna -> anzi che una tripletta per ogni elemento diverso da zero, si salva una coppia e in più un numero per ogni colonna\\
Come risolve il sistema lineare? Uguale a Matlab a meno che il sistema non sia singolare, over o under-determined (ma non è il nostro caso penso)\\
Come tratta le matrici definite positive e quelle generiche? Vedi comportamento di Matlab\\

Octave usa meno RAM di matlab → documentarsi
\end{frame}

\begin{frame}
\frametitle{Librerie usate - Python (Numpy/Scipy)}
Scipy \'e documentato peggio di Matlab ma meglio di Octave \\
Come salva le matrici sparse? Numpy/Scipy si salva le matrici sparse solo con gli indici e il valore degli elementi diversi da zero\\
Come risolve il sistema lineare?
For solving the matrix expression AX = B, this solver assumes the resulting matrix X is sparse, as is often the case for very sparse inputs. If the resulting X is dense, the construction of this sparse result will be relatively expensive. In that case, consider converting A to a dense matrix and using scipy.linalg.solve or its variants.\\
Usa SEMPRE la fattorizzazione LU\\
Come tratta le matrici definite positive e quelle generiche? Non importa perchè usa sempre LU, che va bene per tutte → solve invece usa Cholesky?? C’\'e un modo per usare Cholesky in Scipy??
\end{frame}

%------------------------------------------------

\frametitle{Come sono le matrici?}
\begin{frame}
Fare tabella\\
$ifiss_mat$ non simmetrica, non definita positiva\\
$TSC_OPF_1047$ simmetrica, non definita positiva\\
$ns3Da$ non simmetrica, non definita positiva\\
$GT01R$ non simmetrica, non definita positiva\\
Le altre simmetriche e definite positive, candidate per Cholesky\\
I determinanti? Troppo tempo per calcolarli
\end{frame}

\frametitle{Grafici}
\begin{frame}
Confronto tra Matlab,  Octave, Python su Windows o Linux per i parametri velocità, precisione, occupazione di memoria
\end{frame}

\frametitle{Grafici riassuntivi di confronto}
\begin{frame}

\end{frame}

\frametitle{Listato}
\begin{frame}

\end{frame}

\section{JPEG}


%-----------------------------------------------
% ESEMPI

\begin{frame}
\frametitle{Blocks of Highlighted Text}
\begin{block}{Block 1}
Lorem ipsum dolor sit amet, consectetur adipiscing elit. Integer lectus nisl, ultricies in feugiat rutrum, porttitor sit amet augue. Aliquam ut tortor mauris. Sed volutpat ante purus, quis accumsan dolor.
\end{block}

\begin{block}{Block 2}
Pellentesque sed tellus purus. Class aptent taciti sociosqu ad litora torquent per conubia nostra, per inceptos himenaeos. Vestibulum quis magna at risus dictum tempor eu vitae velit.
\end{block}

\begin{block}{Block 3}
Suspendisse tincidunt sagittis gravida. Curabitur condimentum, enim sed venenatis rutrum, ipsum neque consectetur orci, sed blandit justo nisi ac lacus.
\end{block}
\end{frame}

%------------------------------------------------

\begin{frame}
\frametitle{Multiple Columns}
\begin{columns}[c] % The "c" option specifies centered vertical alignment while the "t" option is used for top vertical alignment

\column{.45\textwidth} % Left column and width
\textbf{Heading}
\begin{enumerate}
\item Statement
\item Explanation
\item Example
\end{enumerate}

\column{.5\textwidth} % Right column and width
Lorem ipsum dolor sit amet, consectetur adipiscing elit. Integer lectus nisl, ultricies in feugiat rutrum, porttitor sit amet augue. Aliquam ut tortor mauris. Sed volutpat ante purus, quis accumsan dolor.

\end{columns}
\end{frame}

%------------------------------------------------

\begin{frame}
\frametitle{Table}
\begin{table}
\begin{tabular}{l l l}
\toprule
\textbf{Treatments} & \textbf{Response 1} & \textbf{Response 2}\\
\midrule
Treatment 1 & 0.0003262 & 0.562 \\
Treatment 2 & 0.0015681 & 0.910 \\
Treatment 3 & 0.0009271 & 0.296 \\
\bottomrule
\end{tabular}
\caption{Table caption}
\end{table}
\end{frame}

%------------------------------------------------

\begin{frame}
\frametitle{Theorem}
\begin{theorem}[Mass--energy equivalence]
$E = mc^2$
\end{theorem}
\end{frame}

%------------------------------------------------

\begin{frame}[fragile] % Need to use the fragile option when verbatim is used in the slide
\frametitle{Verbatim}
\begin{example}[Theorem Slide Code]
\begin{verbatim}
\begin{frame}
\frametitle{Theorem}
\begin{theorem}[Mass--energy equivalence]
$E = mc^2$
\end{theorem}
\end{frame}\end{verbatim}
\end{example}
\end{frame}

%------------------------------------------------

\begin{frame}
\frametitle{Figure}
Uncomment the code on this slide to include your own image from the same directory as the template .TeX file.
%\begin{figure}
%\includegraphics[width=0.8\linewidth]{test}
%\end{figure}
\end{frame}

%------------------------------------------------

\begin{frame}[fragile] % Need to use the fragile option when verbatim is used in the slide
\frametitle{Citation}
An example of the \verb|\cite| command to cite within the presentation:\\~

This statement requires citation \cite{p1}.
\end{frame}

%------------------------------------------------

\begin{frame}
\frametitle{References}
\footnotesize{
\begin{thebibliography}{99} % Beamer does not support BibTeX so references must be inserted manually as below
\bibitem[Smith, 2012]{p1} John Smith (2012)
\newblock Title of the publication
\newblock \emph{Journal Name} 12(3), 45 -- 678.
\end{thebibliography}
}
\end{frame}

%------------------------------------------------

\begin{frame}
\Huge{\centerline{Grazie per l'attenzione}}
\end{frame}

%----------------------------------------------------------------------------------------

\end{document} 
